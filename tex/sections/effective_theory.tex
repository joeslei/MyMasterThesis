\section{リパラメトリゼーションの有効理論 \label{sec:effective_theory}}
この節ではSYK模型の有効理論を扱う。
これによって第\ref{sec:fourpointfunc}節で論じた4点関数の計算に対して第2の視点を得る。

有効理論の作用は\eqref{eq:effectiveActionWithGeneral_q}式で与えられるが、
ここでもう一度掲載する:
\begin{align}
	\frac{S}{N}
	= -\frac{1}{2}\log\det\left(\pdiff{t} - \tilde{\Sigma}\right)
		+ \frac{1}{2}\int dt_1dt_2
		\left[
			\tilde{\Sigma}(t_1, t_2)\tilde{G}(t_1, t_2)
			- \frac{J^2}{q}\tilde{G}(t_1, t_2)^q
		\right].
	\label{eq:effective_theory_of_reparametrization}
\end{align}
ここで$\tilde{\Sigma},\tilde{G}$は経路積分の積分変数である。
古典解は$\Sigma, G$と表記する。
フェルミオンの4点関数は
\begin{align}
	\frac{1}{N^2}\sum_{i,j}\average{\psi_i(t_1)\psi_i(t_2)\psi_j(t_3)\psi_j(t_4)}
	= \int \mathcal{D}\tilde{\Sigma}\mathcal{D}\tilde{G}\ 
		e^{-S}\tilde{G}(t_1, t_2)\tilde{G}(t_3, t_4)
	\label{eq:fourpointfunc_in_G_and_Sigma}
\end{align}
と書き表される。
\eqref{eq:effective_theory_of_reparametrization}式の停留点は
シュウィンガー・ダイソン方程式の解$G,\Sigma$で与えられる。
この解からの揺らぎを計算するならば、その揺らぎをそれぞれ
$\tilde{G} = G + |G|^{\frac{2-q}{2}}g$、
$\tilde{\Sigma} = \Sigma + |G|^{\frac{2-q}{2}}\sigma$とするのが便利である。
この様にしても測度は不変である: 
$\mathcal{D}\tilde{G}\mathcal{D}\tilde{\Sigma} = \mathcal{D}g\mathcal{D}\sigma$。
作用を$g,\sigma$の2次まで展開し、シュウィンガー・ダイソン方程式$G = (\partial_t - \Sigma)^{-1}$
を用いると、揺らぎ$g,\sigma$の作用を得る:
\begin{align}
	\frac{S}{N}
		= -\frac{1}{4J^2(q-1)}&\int dt_1\cdots dt_4\ 
			\sigma(t_1, t_2)\tilde{K}(t_1, \cdots, t_4)\sigma(t_3, t_4)\nonumber\\
		&+ \frac{1}{2}\int dt_1dt_2\ 
		\left(
			g(t_1, t_2)\sigma(t_1, t_2) - \frac{1}{2}J^2(q-1)g(t_1, t_2)^2
		\right).
\end{align}
$\tilde{K}$は\eqref{eq:symmetric_K}式で与えられる対称化したラダーダイアグラムの積分核である。
上式はさらに$\sigma$を積分して消去する事ができ、$g$の作用
\begin{align}
	\frac{S}{N} = \frac{J^2(q-1)}{4}g\cdot(\tilde{K}^{-1} - 1)\cdot g
	\label{eq:effective_action_of_g}
\end{align}
を与える。
4点関数\eqref{eq:fourpointfunc_in_G_and_Sigma}式の被積分関数の$\tilde{G}$を
$|G|^{\frac{q-2}{2}}$に置き換え、\eqref{eq:effective_action_of_g}式を代入して
ガウス積分を実行する事で$O(N^{-1})$の項を計算できる。
結果は\eqref{eq:fourpointfunc_in_theta}式となる。

ここまでの議論は任意のエネルギーで成り立つ。
低エネルギー極限を取り、共形古典解$G_c,\Sigma_c$を使い、
さらに揺らぎとして\eqref{eq:reparameterization_fomula}式で与えられる
リパラメトリゼーション$\delta_{\epsilon}G_c$を選べば、$\delta_{\epsilon}G_c$自身が
$\tilde{K}_c$の固有値1に対する固有関数である事から、
揺らぎの作用\eqref{eq:effective_action_of_g}式は消滅する。
これは\eqref{eq:effective_theory_of_reparametrization}式が
行列式の中の$\partial_t$を捨てる事で、
リパラメトリゼーション$t\to f(t)$及び\eqref{eq:reparametrization_of_G_and_Sigma}式の下で
不変になる事の帰結である。
このリパラメトリゼーション不変性は、共形古典解$G_c$が部分群$SL(2,\mathbb{R})$対称性
しか持たないために自発的に破れる。
$f(t)$はそれによって現れるゴールドストーンボソンである。

共形極限から離れるとリパラメトリゼーション不変性は陽に破れ、
$h=2$の固有値は1から\eqref{eq:corrected_eigenvalue}式に修正され、
その結果ゼロモードに対する有限の作用を得る。
これを計算するために微小変換$t\to t + \epsilon(t)$を考え、$\delta_{\epsilon}G_c$の作用を評価すれば、
\begin{align}
	\frac{S}{N} = \frac{\alpha_S}{\mathcal{J}}\int_0^{\beta}dt\ \frac{1}{2}
		\left[
			\left(\epsilon''\right)^2 - \left(\frac{2\pi}{\beta}\right)^2(\epsilon')^2
		\right],\hspace{20pt}
	\alpha_S \equiv \frac{\alpha_K}{6q^2\alpha_0} = \frac{q|k'_c(2)|\alpha_G}{6q^2\alpha_0}
	\label{eq:action_for_epsilon}
\end{align}
という有限の作用を得る。
従って共形対称性は自発的にも、また陽にも破れる。
次に\eqref{eq:action_for_epsilon}式に有限のリパラメトリゼーション$t\to f(t)$を施した表式を求めたい。
これには零温度から始めるのが良い。
従って$t$や$f$は直線上の座標となる。
$f(t)$はマクローリン展開すれば
\begin{align}
	f(t) = f(0) + f'(0)\left(t
		+ \frac{1}{2}\frac{f''(0)}{f'(0)}t^2 + \cdots
	\right)
\end{align}
となり、小さい$t$に関しては、スケーリングや並進によって
$\epsilon' = 0, \epsilon'' = f''/f'$となるような微小変換を得る。
このスケーリングや並進は零温度2点関数に影響しないため、
次式のような微小変換から有限の変換への一般化が可能である:
\begin{align}
	\frac{1}{2}\int dt\ (\epsilon'')^2 \to 
	\frac{1}{2}\int dt\ \left(\frac{f''}{f'}\right)^2.
\end{align}
これによって作用\eqref{eq:action_for_epsilon}式は
\begin{align}
	S = -N\frac{\alpha_S}{\mathcal{J}}\int dt\ \{f, t\},\hspace{20pt}
	\{f, t\} = \frac{f'''}{f'} - \frac{3}{2}\left(\frac{f''}{f'}\right)
	\label{eq:Schwartzian_action}
\end{align}
となる。$\{f,t\}$はシュワルツ微分であり、
$SL(2, \mathbb{R})$変換$f \to \frac{af + b}{cf + d}$の下で不変という性質を持つ。
零温度では$G_c$が$SL(2, \mathbb{R})$不変なので、この対称性は厳密なものである。
有限温度に移るには
\begin{align}
	f(t) = \tan\left(\frac{\pi t}{\beta}\right)
\end{align}
という変換をすればよい。

\pagebreak