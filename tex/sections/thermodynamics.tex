\section{SYK模型の熱力学的諸量\label{sec:thermodynamics}}
この節ではSYK模型の自由エネルギーやエントロピーなどの熱力学的な量を計算する. 
特に零温度エントロピーや比熱は後に論ずるSYK模型の量子カオス的性質に関わってくる. 

出発点は\eqref{eq:effective_theory_of_reparametrization}式であり, 
自由エネルギーは
\begin{align}
	e^{-\beta F} = \int \mathcal{D}\tilde{G}\mathcal{D}\tilde{\Sigma}\ 
		e^{-S[\tilde{G}, \tilde{\Sigma}]}
\end{align}
より与えられる. 
特にこれを低温度で展開すると
\begin{align}
	-\frac{\beta F}{N}
	= -\beta E_0 + S_0 + \frac{c}{2\beta} + O\left(\frac{1}{\beta^2}\right)
\end{align}
となる. 
ここで$E_0$は基底エネルギー, $S_0$は零温度エントロピー, そして$c/\beta$は比熱である. 
$E_0$は自由エネルギーに$J\partial_J$を作用させる事で得られる:
\begin{align}
	J\frac{\partial}{\partial J}\left(-\frac{\beta F}{N}\right)
	= \frac{J^2\beta}{q}\int_0^{\beta}dt\ G(t)^q
	= -\frac{\beta}{q}\left.\frac{\partial G}{\partial t}\right|_{t\to 0^+}
	= \beta E.
\end{align}
ここで自由エネルギーを与える古典解$G$, $\Sigma$は運動方程式に従う事から, 
$\partial_J$は$J$に陽に依る項のみに作用するという事を用いた. 
零温度エントロピーは一般の$q$において
\begin{align}
	\frac{S_0}{N}
	= \frac{1}{2}\log 2 - \int_0^{1/q} dx\ \pi\left(\frac{1}{2} - x\right)\tan \pi x
\end{align}
で与えられる. 
また比熱は\eqref{eq:Schwartzian_action}式のシュワルツ理論を有限温度にしたものから得られる. 
具体的にはパラメータ$t$に$f(t) = \tan(\pi t / \beta)$のようにリパラメトリゼーションを施し, 
自由エネルギーへの有限温度の補正を計算すると
\begin{align}
	-\beta F \supset \frac{N \alpha_S}{\mathcal{J}}
		\int_0^{\beta} dt\ \left\{\tan \frac{\pi t}{\beta}, t\right\}
	= 2\pi^2\alpha_S\frac{N}{\beta \mathcal{J}}
\end{align}
を得る. 
ここで$\alpha_S$は\eqref{eq:action_for_epsilon}式で与えられるシュワルツ作用の係数である. 
この補正項より比熱は
\begin{align}
	\frac{c}{2} = 2\pi^2\alpha_S\frac{N}{\mathcal{J}}
	\label{eq:specific_heat}
\end{align}
から与えられる. 

\pagebreak